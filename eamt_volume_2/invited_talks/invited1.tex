Machine translation (MT) is an essential tool for one of the largest institutional translation providers in the world: the European Parliament’s Directorate-General for Translation (DG TRAD). DG TRAD is home to 24 language units that embody and put into practice one of the core democratic principles of the European Union: multilingualism. In this complex environment, MT has become an integral part of DG TRAD’s work, helping it to manage an ever-growing volume of translation requests and allowing it to focus on the unique value that only humans can bring to the translation process. 

The MT technology used in DG TRAD is a focal point of cooperation between the EU institutions and is constantly evolving. To best harness the benefits, DG TRAD relies on a dedicated team that carries out tests to explore the best ways of using MT for DG TRAD’s content. 

This presentation will tell you, from a user’s perspective, about DG TRAD’s journey to identify the most efficient ways of working with MT. Here are some of the questions we will cover:

\begin{itemize}
\item How well does MT handle the European Parliament’s content? Do all languages produce the same results? How does MT quality vary based on the type of content?
\item How does MT improve efficiency? What efforts are still necessary after integrating MT into DG TRAD’s workflow?
\item What about clear language? How well does MT perform in this area?
\end{itemize}

Finally, we will look at the new areas DG TRAD is exploring in this age of artificial intelligence (AI) and where we see that further research could provide added value.


