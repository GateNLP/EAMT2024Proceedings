Helena Moniz is the President of the European Association for Machine Translation (2021-) and President of the International Association for Machine Translation (2023-). She is also the Vice-Coordinator of the Human Language Technologies Lab at INESC-ID, Lisbon. Helena is an Assistant Professor at the School of Arts and Humanities at the University of Lisbon, where she teaches Computational Linguistics, Computer Assisted Translation, and Machine Translation Systems and Post-editing. She is now in a very exciting project, coordinated by Unbabel, the Center for Responsible AI (https://centerforresponsible.ai), within the Portuguese Recovery and Resilience Plan, as Chair of the Ethics Committee.
Helena graduated in Modern Languages and Literature at the School of Arts and Humanities, University of Lisbon (FLUL), in 1998. She took a Teacher Training graduation course in 2000, a Master’s degree in Linguistics in 2007, and a PhD in Linguistics at FLUL in cooperation with the Technical University of Lisbon (IST) in 2013. She has been working at INESC-ID/CLUL since 2000, in several national and international projects involving multidisciplinary teams of linguists and speech processing engineers. Within these fruitful collaborations, she participated in more than 20 national and international projects.
From 2015/09 to 2024/04, she was the PI of a bilateral project between INESC-ID and Unbabel, a translation company combining AI + post-editing, working on scalable Linguistic Quality Assurance processes for crowdsourcing. She was responsible for the implementation in 2015 of the MQM metric, the creation of the Linguistic Quality Assurance processes developed at Unbabel for Linguistic Annotation and Editors' Evaluation. She also worked on research projects, involving Linguistics, Translation, and Responsible AI, and products developed by the Labs Team, mostly cultural transcreation, high risk products, and silently controlled language metrics for dialogues.
In a sentence, she is passionate about Language Technologies in a human-centric perspective and always feels like a child eager to learn!
