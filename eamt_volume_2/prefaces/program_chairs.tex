On behalf of the programme chairs, a warm welcome to the 25th annual conference of the European Association for Machine Translation in Sheffield, UK. Following last year’s restructuring of the research track into two tracks, this year’s conference programme is divided into four tracks, two dedicated to research (one for technical papers for development of MT techniques and one focused on translators and users of MT), an implementations and case studies track and a projects and products track.
\\

The {\bf Technical Research track} invited submissions on significant results in any aspect of MT and related areas, including multilingual technologies. As in previous years, this track proved the most popular of the four tracks, receiving a total of 46 submissions from 26 different countries. With one desk rejection and four paper withdrawals, 20 papers were accepted from 18 different countries, resulting in an acceptance rate of 43\%, which is consistent with previous years. Six of the accepted papers are to be presented orally and the remaining 14 will be presented as posters.
\\

Following current practices in the field, papers focus on neural MT (NMT), with several works also studying large language models (LLMs) for translation. Accepted papers represented a wide range of topics relevant to current interests in the field: context-aware MT (Appicharla et al., 2024; Gete and Etchegoyhen, 2024); the application of techniques for low-resource languages and scenarios (Chen et al. 2024; Guttmann et al.; Simonsen and Einarsson, 2024; Song et al. 2024) including sign language translation (McGill et al., 2024); attention to specific domains (Ploeger et al., 2024; Roussis et al. 2024)  and to the challenges faced when dealing with them, e.g. for the incorporating of terminologies (Hauhio and Friberg. 2024). A number of works study LLMs (Chen at al, 2024.; Mujadia et al. 2024; Simonsen and Einarsson, 2024), a trend that is likely to continue in years to come. As a sign of the progress being made in the quality of MT systems, the EAMT 2024 technical research track also features several papers dealing with topics related to the alignment of MT outputs with the expectations of human users (Moura Ramos et al.,  2024), including on the topics of toxicity (García Gilabert et al., 2024), formality (Wisniewski et al., 2024) and gender-inclusiveness (Piergentilie et al., 2024).
\\

We would like to give our thanks to all the authors who submitted to the track and to the 72 reviewers, who provided feedback and insightful comments for the submissions received. We are particularly grateful to the emergency reviewers who agreed to review papers at the last minute in order for decision notifications to be sent out on time.
\\


{\bf Translators and Users Track}
\\

The focus of the Translators and Users track is to cover a wide range of topics related to the interaction between human translators and other users of machine translation. The second edition of this track attracted 21 papers, with 18 accepted out of them which comprises 85.71\% of acceptance. Five of the accepted papers will be presented orally and 13 will be presented at a dedicated poster presentation session. The accepted papers address the interaction between machine translation and its users from various perspectives and cover various aspects of machine translation use, including both interlingual and intralingual translation, looking into challenges and potentials of large language models, as well as correlating human and machine translation. They provide novel examinations of long-standing areas of interest for translators and users in this space including translation quality, MT performance, tools and methods to assist translators, and users’ perceptions and attitudes towards MT.
\\

Sui He experiments with prompts applying ChatGPT for automatic translation. The author compares translation briefs and what s/he calls persona prompts (assignment of a role of an author or translator to the system). 
\\

Claudio Fantinuoli and Xiaoman Wang explore correlation between automatic quality evaluation metrics with human judgements for simultaneous interpreting. 
\\

Serge Gladkoff et al. investigate the application of the state-of-the-art LLMs for uncertainty estimation of MT output quality, which is required to determine the need for post-editing.
\\

Paolo Canavese and Patrick Cadwell analyse translators’ perspectives on the use of machine translation and its impact in a specific institutional setting, i.e. the Swiss Confederation.
\\

Marta R. Costa-jussà et. al. presents a novel multimodal and multilingual pipeline to automatically identify and mitigate added toxicity at inference time, which does not require further model training.
\\

Celia Soler Uguet et al. compare performance of various LLMs for automatic post-editing and MQM error annotation across four languages in a medical domain.
\\

Lise Volkart and Pierrette Bouillon compare human translation and post-edited machine translation from a lexical and syntactic perspective in two language pairs: English-French ad German-French. Their aim is to find out if NMT systems produce lexically and syntactically poorer translations.
\\

Gabriela Gonzalez-Saez et al. describe their work  on visualisation tools to foster collaborations between translators and computational scientists. 
\\

Maria Kunilovskaya et al. explore if GPT-4 can reduce translationese (specific feature of translated texts) in human-translated texts on bidirectional German-English data from the Europarl corpus.
\\

Rachel Bawden et al. evaluate the effectiveness of a post-editing pipeline for the translation of scientific abstract demonstrating that such pipelines can be effective for high-resource language pairs.
\\

Vicent Briva-Iglesias and Sharon O'Brien present a user study on professional English-Spanish translators in the legal domain, which focuses on impact of negative or positive translators’ pre-task perceptions of MT.
\\

Miguel Rios et al. explore the impact of automatic speech synthesis in a post-editing machine translation environment in terms of quality, productivity, and cognitive effort.
\\

Silvana Deilen et al. evaluate performance of intralingual machine translation systems in the area of health communication.
\\

Michael Carl looks into a way of using  machine learning to validate the empirical objectivity of a taxonomy for behavioral translation data.
\\

João Lucas Cavalheiro Camargo et al. conduct a survey aimed at identifying and exploring the attitudes and recommendations of machine translation quality assessment educators.
\\

Bettina Hiebl and Dagmar Gromann propose to use the Best-Worst scoring for a comparative translation quality assessment of one human and three machine translations in the English-German language pair.
\\

Adaeze Ngozi Ohuoba et al. investigate methods to detect critical and harmful MT errors caused by non-compositional multi-word expressions and polysemy. For this, they design diagnostic tests that they apply on collections of medical texts.
\\

Nora Aranberri explores evaluation of the Spanish-Basque translations. The author compares evaluations done by volunteers and translation professionals.
\\

We would like to thank the 28 colleagues that kindly gave their time and effort to review the papers submitted to this track. Your reviews were perceptive, detailed, and, above all, constructive. We would also like to express our special gratitude to those reviewers who stepped in at the last minute to provide extra reviews at short notice. Your collegiality was a great support to us.
\\

{\bf Implementations and case studies track} 
\\

Entering the second year with the Implementations \& Case Studies track, we are excited to share the acceptance of 9 papers. These papers cover a wide range of topics, showing the latest advancements, challenges, and creative ideas in MT. The goal for this track remains unchanged: to report experiences with MT in organizations of all types (both industry and academia) and to share views and observations based on day-to-day experiences working within the dynamic field of MT.
\\

The journey begins with Oliver et al. who detail corpus creation and NMT model training for legal texts in low-resource languages, shedding light on the intricacies of bridging linguistic gaps in specialized domains.
\\

Continuing on this path, Eschbach-Dymanus et al. delve into the realm of domain adaptation of MT for business IT texts, offering valuable insights into the translation capabilities of LLMs.
\\

Bechara et al. present the creation and evaluation of a multilingual corpus of UN General Assembly debates, underscoring the importance of robust linguistic resources in advancing our understanding of multilingual communication.
\\

Additionally, Korotkova and Fishel present groundbreaking research on Estonian-centric MT, emphasizing data availability and releasing a back-translation corpus of over 2 billion sentence pairs.
\\

Moving forward, Silveira et al. examine the suitability of GPT-4 in generating subject-matter expertise assessment questions, illuminating new avenues for leveraging artificial intelligence in language assessment.
\\

Continuing in this direction, Nunziatini et al.'s research explores the advantages and disadvantages of using LLMs to make raw MT output gender-inclusive.
\\

Berger et al. work in prompting LLMs with human error markings represents a significant step towards self-correcting MT, offering promising avenues for enhancing translation quality in specialized domains.
\\

Vasiļjevs et al. present findings from a comprehensive market study on advancing digital language equality in Europe. They provide critical insights into the current landscape of multilingual website translation and introduce innovative open-source solutions aimed at bridging linguistic divides.
\\

Lastly, Vincent et al. present an insightful case study on contextual MT in professional subtitling. This work sheds light on the practical implications of incorporating extra-textual context into the MT pipeline, offering valuable lessons for industry practitioners.
\\

Together, these papers paint a vivid picture of the ever-evolving landscape of MT Implementations \& Case Studies, showcasing the ingenuity, resilience, and collaborative spirit of the MT community.
\\

{\bf Products and Projects track} 
\\

This year we received 31 submissions and 30 papers were accepted. The selection will provide a plethora of products and projects being developed by our community with a rich set of topics, ranging from EAMT sponsored projects, European projects, services and products from distinguished industry and research players of our community. It will surely be a very lively session with the usual poster boasters (one of our EAMT conferences’ favourite moments) and poster sessions. We would like to thank the 25 reviewers, who were drafted quite late, for their quick response and their timeliness.

\begin{center}
\begin{tabular}{ c c c}
	\makecell{Rachel Bawden \\ (Inria, Paris, France) \\ \vspace{0.5cm}} & \makecell{Víctor M Sánchez-Cartagena \\ (University of Alacant, Spain) \\ \vspace{0.5cm}} & \makecell{Patrick Cadwell \\ (DCU, Ireland) \\ \vspace{0.5cm}} \\ 
	\makecell{Ekaterina Lapshinova-Koltunski \\ (University of Hildesheim, Germany)\\ \vspace{0.5cm}} & \makecell{Vera Cabarrão \\ (Unbabel, Portugal)\\ \vspace{0.5cm}} & \makecell{Konstantinos Chatzitheodorou \\ (Strategic Agenda, UK)\\ \vspace{0.5cm}} \\
	\makecell{Helena Moniz \\ (University of Lisbon (FLUL) \\ INESC-ID, Portugal)\\ \vspace{0.5cm}} & \makecell{Mikel Forcada \\ (Prompsit Language Engineering \\ Elx, Spain)\\ \vspace{0.5cm}} & \makecell{Mary Nurminen \\ (Tampere University, Finland)\\ \vspace{0.5cm}} \\
	\makecell{Diptesh Kanojia \\ (University of Surrey, UK)} & \makecell{Barry Haddow \\ (University of Edinburgh, UK)} &  \\

\end{tabular}
\end{center}



