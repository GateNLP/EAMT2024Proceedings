For the 2023 best thesis award, we received a total of 9 submissions; all were MT-related thesis defended in 2023.  We recruited 20 reviewers to examine and score the theses, considering how challenging the problem tackled in each thesis was, how relevant the results were for machine translation as a field, and what the strength of its impact in terms of scientific publications was. Two EAMT Executive Committee members also analysed all theses. It became very clear that 2023 was another very good year for PhD theses in machine translation. 
\\

All theses had merit, all candidates had strong CVs and, therefore, it was very difficult to select a winner.
\\

A panel of two EAMT Executive Committee members (Barry Haddow and Helena Moniz) was assembled to process the reviews and select a winner that was later ratified by the EAMT executive committee.
\\

We are pleased to announce that the {\bf winner of the 2023 edition of the EAMT Best Thesis Award} is {\bf Marco Gaido's' thesis ``Direct Speech Translation Toward High-Quality, Inclusive, and Augmented Systems''} (FBK, Italy ), supervised by Marco Turchi and Matteo Negri.
\\

In addition, the committee judged  that the following theses, were {\bf ``highly commended''}:
\\
{\bf Jannis Vamvas}: “Model-based Evaluation of Multilinguality” (University of Zurich, Switzerland), supervised by Rico Sennrich and Lena A. Jäger
\\

{\bf Javier Iranzo-Sánchez}: ``Streaming Neural Speech Translation'' ( UPV, Spain), supervised by Jorge Civera and Alfons Juan 
\\

The awardee will receive a prize of €500, together with a suitably-inscribed certificate. In addition, Dr. Gaido will present a summary of their thesis at the 25th Annual Conference of the European Association for Machine Translation.  In order to facilitate this, the EAMT will waive the winner's registration costs, and will make available a travel bursary of €200.


\bigbreak

Barry Haddow, chair, EAMT BTA award 2023 \\
University of Edinburgh, UK
