\textit{Ey Up!}
\\

We are delighted to welcome you to EAMT2024 at Sheffield and celebrate its 25th anniversary. Sheffield, renowned for its rich industrial heritage and pivotal role in the steel industry, provides an ideal venue for “forging” collaboration and exchanging ideas. "The outdoor city" provides an ideal and welcoming environment for a thriving international community with a large number of students. "The UK's greenest city" has the Peak District National park at its doorstep, being a not to be missed place for the most adventurous (looking for sports like bouldering and mountain bike) as well as for just relaxing in a short walk enjoying the views and hospitality of the park's small villages. It is not rare that students end up staying in Sheffield and calling this fabulous place home (which is the case of some of us in the organising committee). 
\\

The University of Sheffield has also been key in developing Machine Translation research, being an active member of EAMT and part of its history. Memorable former members of the Sheffield community include: the late John Hutchins (creator of the MT Archive and author of the 1992 book An introduction to machine translation) was a librarian in Sheffield from 1965 and 1971; the late Professor Yorick Wilks (author of the 2008 book Machine Translation: Its Scope and Limits) was an emeritus professor and a former head of the Computer Science department; and Professor Lucia Specia (the pioneer in the area of MT Quality Estimation and author of the 2018 book Quality Estimation for Machine Translation) was professor at the Computer Science department and former PhD supervisor of two of the local organisers. 
ZOO Digital is a globalisation company being a pioneer in the field of entertainment. They provide localisation services for tasks such as dubbing, subtitling, audio post-production, audio description, scripting, among others. ZOO is a long term partner of the University of Sheffield, being committed to support research in speech and text translation. They are also one of the most active sponsors of our Center for Doctoral Training (CDT) and had their first sponsored PhD student working on the area of MT graduating in 2023. 
\\

We are especially excited about our conference venues, which showcase some of Sheffield’s most iconic sites. Our welcome reception will take place in the stunning Sheffield Winter Garden, one of the largest temperate glasshouses in the UK. This beautiful indoor garden is filled with exotic plants from around the world. The conference dinner will be hosted at the Kelham Island Museum, a celebrated institution that chronicles the city’s industrial history and innovation in steel production. Attendees will have the unique opportunity to visit the impressive River Don Engine, a steam engine that highlights Sheffield’s engineering and industrial heritage. We are also thrilled to announce that ZOO Digital has generously funded a special pre-conference social event at the National Videogame Museum. This interactive museum celebrates the history and culture of video games, offering a fun and engaging way for attendees to unwind and connect with each other. Finally, participants that opt to attend the Kelham Island Food tour will be taken on a culinary journey of the area, visiting a range of eating establishments and enjoying generous samples at each stop, and gaining insight into the interesting history of this famous Sheffield district.
\\

We extend our deepest gratitude to our Silver Sponsors (Language Weaver, Translated, Unbabel), Bronze Sponsors (AppTek, CrossLang, Pangeanic, STAR Group, TransPerfect), Collaborator (Apertium), Supporter (Springer Nature), Media Sponsors (MultiLingual), Program Chairs (Helena Moniz, Rachel Bawden, Víctor M Sánchez-Cartagena, Patrick Cadwell, Ekaterina Lapshinova-Koltunski, Vera Cabarrão, Konstantinos Chatzitheodorou, Mikel Forcada, Mary Nurminen, Diptesh Kanojia, Barry Haddow), keynote speakers (Alexandra Birch, Valter Mavrič), and the program committee, and authors.
\\

Our special very thanks goes to the volunteers (Freddy Heppell, Tom Pickard, Edward Gow-Smith, and Shenbin Qian), administrative support (Natalie Hothersall and Kim Matthews-Hyde), events management (Gavin Lambert), and our emergency organisation support committee (Xi Wang and Mark Stevenson) whose hard work and dedication have made this conference possible. We also thank the head of the Computer Science department, Professor Heidi Christensen, for her support to our conference. 
\\

Finally, we invite you to explore and enjoy the city of Sheffield. Whether you are discovering its historical landmarks, enjoying its green spaces, or immersing yourself in its rich cultural offerings, we hope you find inspiration both within and beyond the conference sessions.
\\

\begin{center}
\begin{tabular}{ c c c}
	\makecell{Carolina Scarton \\ (University of Sheffield) \\ (EAMT Secretary) \\ \vspace{0.5cm}} & \makecell{Charlotte Prescott \\ (ZOO Digital) \\ \\ \vspace{0.5cm}} & \makecell{Chris Bayliss \\ (ZOO Digital) \\ \\ \vspace{0.5cm}} \\ 
	\makecell{Chris Oakley \\ (ZOO Digital)\\ \vspace{0.5cm}} & \makecell{Joanna Wright \\ (University of Sheffield)\\ \vspace{0.5cm}} & \makecell{Stuart Wrigley \\ (University of Sheffield)\\ \vspace{0.5cm}} \\
	& \makecell{Xingyi Song \\ (University of Sheffield)} &   \\

\end{tabular}
\end{center}



