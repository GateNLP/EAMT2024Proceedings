LLMs such as ChatGPT, Claude and Gemini 1.5 have come to dominate the AI landscape, through their ability to perform well across a wide range of tasks and languages. They have excellent abilities in machine translation for high-resource languages, often performing on par with dedicated translation models, and with exciting use-cases including stylization, post-editing, and human-in-the-loop approaches. Nevertheless, these models’ capabilities are much more limited in languages with less digital representation: performance in lower-resource languages can be regarded as a byproduct rather than a focus and the reliance on English language training data reinforces English language cultural hegemony, with particularly high representation of American English cultural knowledge in model weights. In downstream evaluation, claims of multilinguality typically belie the dependence on English-centric data: the FLORES dataset, for example, which contains MT evaluation data in over 200 languages, is largely translated from English. This panel will explore the challenges and opportunities associated with LLMs for translating low-resource languages, investigating the dangers of exacerbating existing linguistic and cultural biases, the potential of LLMs to democratise information access, and how to ensure that these models benefit rather than marginalise underrepresented linguistic communities.
